%!TEX root = ../thesis.tex
%*******************************************************************************
%*********************************** First Chapter *****************************
%*******************************************************************************

\chapter{Introduction}  %Title of the First Chapter
\section{Objectives}
Social media expands the data available to mental health clinicians. On the one hand, social media platforms such as Facebook and Twitter provide longitudinal data of a person's daily life activities and interests. Longitudinal data is especially valuable to the analysis of affective disorders as it enables clinicians to monitor the change of valence. On the other hand, anonymous platforms such as Reddit encourages users to disclose experience of sensitive topics (e.g. suicide), which allow clinicians to monitor sensitive life events in the community. It provides clinicians with a new platform to conduct analysis on the recovery process and treatment planning.

The current social media and mental health studies mainly focus on predicting types and severeness of mental disorders, predicting behavioral changes among those who struggle with mental disorders, forecasting psychological traits, monitoring mood changes, inspecting social support network and behaviors. Some of these studies are based on the intuition of how psychopathology is assessed but not the actual clinical assessment procedures. To better assist clinical diagnosis and treatment planning, our studies include the diagnosis framework, the needs of the patients and the needs of the clinicians as the purpose of our studies. 

Despite the fact that social media better equips and better informs the mental health research, there are limitations of using social media data to inspect mental health issues. One of the limitations is that the information we gain from social media only involves individuals who are keen at using social media, a second limitation is that the content posted on the platform is susceptible to selective bias. It is important to be aware of these limitations while conducting treatment planning because people who need support most might not be in the sample of interests.


Having the limitations bear in mind, my PhD projects will combine simple natural language processing techniques with machine learning to understand how individuals reveal their psychological traits, mental disorder symptoms and help seeking behaviours on social media. My PhD projects mainly focus on stress and social support. The choice of my focus is based on the fact that stress is important to the development of mental health disorders and it can be inferred by valence and thoughts. Social media platforms provide us longitudinal data that are indicative of thoughts and valence. Nevertheless, social media platforms are mediums for people to seek for social support. It's important to study how people seek for social support on social media platforms. What type of support they are seeking and the therapeutic effect provided by the audiences. These studies are beneficial to treatment planning and understanding the recovering process. To be specific, my objectives are as follows:

1. Identify and forecast mental health related symptoms using social media data. (predisposing factor, perpetuating factor and presenting problems)
RQ 1: Do valence and thoughts in social media posts indicate mental health vulnerability? (personality, satisfaction with life and self-disclosure)
RQ 2: What are the valence and thought patterns in those who have greater risks in depressive disorders?
RQ 3: What are the characteristics of redditors who show greater distress when disclosing a stressful life event? Do they show greater loneliness and use reddit more often? What communities or topics they often endorse?
By inspecting the pattern of valence system and thoughts, we identify the characteristics of people who have greater mental health vulnerability or risks. Social media platforms provide us longitudinal data to track the change of valence and the existence of cognitive distortions. The change in these components does not contribute to the development of a specific type of mental disorder but contribute to symptoms that occurs across many types of disorders. In the next stage, we further identify several types of valence defined in the research domain criteria (RDoC). 

2. Study how social support is provided and perceived on social media. (protective factor)
RQ 1: What properties of the original post affect the amount of comments received? What types of posts and comments tend to be discussed in depth? 
RQ 2: To what extent do original posters react to the comments, and are there any indications that they perceive that social support has been given?
By inspecting how people disclose a stressful life event on anonymous social media sites. We study what types of social support people are seeking, does the support provided match their needs and do they perceive that they receive support from other social media users. This study can assist clinicians in treatment planning.

3. Inspect and monitor chronic stress on social media. (perpetuating factor and protective factor)
RQ 1: What are language characteristics of people who disclose chronic stress on social media and those who disclose acute stress on social media?
RQ 2: For those who disclose chronic stress on social media, what social support are they seeking?
RQ 3: For those who disclose chronic stress on social media, how do they interact with people having strong or weak social ties with them?
Stressful life events and valence have been widely studied on social media, whereas, chronic stress such as role stress as perpetuating factors are not welled studied yet. By using data from social media supporting group or community, we will analyze how people disclose their chronic stress experience on anonymous social media sites.

4. Investigating cultural differences in disclosing chronic and acute stress on social media.
RQ 1: How does people from Chinese culture disclose chronic and acute stress on social media?
RQ 2: How does social support requested and provided in Chinese social media?

Each objective above is linked to the five factor case formulation model. The five factors include three risk factors (predisposing factors, precipitating factors, perpetuating factors), protective factors and presenting factors. Case formulation describes the symptoms and underlying mechanism of the psychological disorders. In Chapter 2, case formulation and the standard procedures in clinical diagnosis are introduced. We explain the five factor model in case formulation and inspect how researchers use social media data to infer various components in the five factor model. In addition, dozens of mental health related social media studies in the recent 10 years are categorized according to the five factor models.

In the past one year, I have completed a study on the second objective and another study that answer research question 1 and 2 on the first objective is about to be finished. Chapter 2 gives the background of what psychological constructs are related to the development of mental disorders and what constructs can be inferred from social media data. Chapter 3 is a study on the 2 objective, it inspects how social support is provided to redditors who experienced infidelity. A set of posts reporting an experience of being cheated on by a partner was collected in April 2018. Language of the posts and comments, upvotes and downvotes, interaction between the post authors and the audiences were analyzed to identify what factors affect the engagement level of the post authors and the audiences. Chapter 4 answers the first two research questions on the first objective. This study illustrates how to identify cognitive distortion on social media data. We develop an annotation guideline to identify valence and cognitive distortion on Facebook text. Our preliminary results show that cognitive distortion has stronger correlation with participants' CES-D score (Center for Epidemiologic Studies Depression Scale) than valence.  Chapter 5 discusses the future work planning and timeline for my PhD.








