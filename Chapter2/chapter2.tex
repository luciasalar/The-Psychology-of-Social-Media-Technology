%!TEX root = ../thesis.tex
%*******************************************************************************
%****************************** Second Chapter *********************************
%*******************************************************************************

\chapter{Background}

introduction to 4P, 

Traits that are related to psychological disorders
psychological disorder symptoms

\section{Factors contributing to mental health}
% Uncomment this line, when you have siunitx package loaded.
%The SI Units for dynamic viscosity is \si{\newton\second\per\metre\squared}.
The 4P model is a biopsychosocial approach to analyze mental health development and progression \cite{}. The 4P model suggests that there are four types of factors contribute to development of all psychological disorders, each type of factors can be further divided into biological factors, psychological factors and social factors. The four types of factors are: predisposing risk factors, precipitating risk factors, perpetuating risk factors and protective factors. 

Predisposing risk factors make an individual more susceptible to a psychological disorder. For example, genetic component, gender differences, personality traits and parental modeling. Precipitating risk factors contribute to the onset of psychological disorders, for instance, traumatic event or stressful life events. Perpetuating factors maintain the symptoms span from psychological disorders, such as chronic stress, cognitive bias, social stigma, social roles. Finally, protective factors build up the resilience of an individual \cite{}. Figure \ref{tab:4P} demonstrates the relationships of these factors. Note that some of the precipitating factors can be perpetuating factors if they exist for a long time. 

We attempt to study some of the factors via the lenses of social media platform. Social media behaviors reflect psychological and social factors to some extend but it's nearly impossible to inspect biological factors from social media data, except for gender differences. Another factor that is difficult to inspect in social media at this stage is childhood experience. As a result, my PhD project will only focus on inspecting factors that are available in social media data.

% Please add the following required packages to your document preamble:
% \usepackage{graphicx}


% section section_name (end)
\begin{table}
\caption{Biopsychosocial Approaches in 4P Model}
\label{tab:4P}
\resizebox{\textwidth}{!}{%
\begin{tabular}{|l|p{2cm}|p{4cm}|p{4cm}|}
\hline
\textbf{4P factor model}            & \multicolumn{3}{l|}{\textbf{Biopsychosocial approach}}      \\ \hline
                           & \textbf{biological factors} & \textbf{psychological factors}          & \textbf{social factors}                                      \\ \hline
\textbf{Predisposing risk factors}  & genetic, gender    & personality traits, mood stability, self-disclosure, self-esteem, perfectionism, sensitivity to stress       & parental modelling,                                 \\ \hline
\textbf{Precipitating risk factors} & physical injuries  & stress/trauma                                                                                                & change in family dynamics                           \\ \hline
\textbf{Perpetuating risk factors}  & chronic diseases   & chronic stress, malnutrition, crowding, pollution                                                            & social stigma, social roles, culture, social status \\ \hline
\textbf{Protective factors}         & genetic            & effective coping skills, high resilience, high cognition skills, high self-esteem, low sensitivity to stress & social support                                      \\ \hline
\end{tabular}%
}
\end{table}

\subsection{predisposing factors}
When individuals are exposed to a stressor, predisposing factors influence how people experience stress and how they cope with it. These factors affect an individual in the probability to encounter stressors. Note that people select their environment and shape them. They also affect people's tendency to react to situation and their tendency in coping. \cite{} (Suls and Martin, 2005; Vollrath, 2001; cf. Watson, David and  Suls, 1999)

\subsubsection{personality}

Consistent evidence has been found that some dimensions in personality co-occurred with certain mental disorders, but there's no direct link between personality and mental health problems. Personality is largely shaped by child-rearing practices and family interactions (cite). As a result, the link between personality and mental health can be traced back to family rearing style.

There are many types of personality models, such as, Myers–Briggs Type Indicator (MBTI),Cattell’s 16 Factor Model, Big Five Model, among which the five factor personality model (Big5) is most widely adopted in personality studies. Big5 suggests five board dimensions to describe human personality:  openness to experience, conscientiousness, extraversion, agreeableness, and neuroticism. There has been consistent evidence showing that personality plays an important role in the experience of stress. Individuals with high neuroticism often experience more negative emotions (cite), and negative emotions affect how an individual cope with stress \cite{kiecolt2002emotions}. Neurotism has been found to be an increase risk factor for depression and anxiety. \cite{talley1986association,cattell1961meaning} Internet addiction often occurs in introverted people \cite{xiuqin2010mental}. However, personality and mental health do not have a direct causation relationship. There are many mediating factors between personality types and mental disorders. For example, rumination and worrying mediate neuroticism and depression \cite{roelofs2008rumination};

\subsubsection{self-disclosure}
Self-disclosure is the amount of information an individual disclose to significant others. There are three major parameters of self-disclosure: the amount of information disclosed; depth of the information disclosed and the time spent on describing the information. The evaluation of self-disclosure must involve the three parameters \cite{cozby1973self}.


Self-disclosure is important to mental health problem because it affects one's help seeking behaviors. Individuals with high self-disclosure level are more likely to seek professional help (Hinson and Swanson, 1993). Self-disclosure is also a critical problem for psychotherapist. On one hand, patients often go to psychotherapy to seek solution to loneliness, which is a result of difficulty in building intimate relationships. One the other hand, patients wish to conceal themselves \cite{fisher1990shared, stricker1990self}. 

Self-disclosure is, again, shaped by the family rearing style. Fisher(1982) proposes that when infant/child involves in a communication with significant others, they receive positive or negative feedback . The feedback provided form a feedback loop. The positive feedback loop encourage a child to disclose more, whereas, the negative feedback loop does the opposite. Failing to disclose results in an accumulation of secrets, the greater the degree of secrets, the more alienation of self and others. cite() Besides family rearing style, self-disclosure is also affected by gender, culture, race and the relationship with the significant others.

\subsubsection{self-esteem}
Self-esteem is the beliefs and evaluation individuals hold about themselves \cite{burns1982self}. These beliefs are inner influences that guild and government individuals through out their lives. The development of self-esteem is dependent on support and approval from significant others and self-perceived competence in the domains that are important to self. The distance between ideal self and real self is critical to the development of self-esteem cite. 

Self-esteem is a protective factor in mental health because self-esteem and the ability to face challenge determine what happens to an individual in challenging situations. Conversely. poor self-esteem plays a critical role in the development of mental disorders. \cite{mann2004self}. High self-esteem is related to better life satisfaction \cite{zimmerman2000self}.  Self-esteem is a strong predictor of subjective well-being \cite{furnham2000perceived}. 

Whereas, Low self-esteem lead to maladjustment \cite{garmezy1984study}, depressed mood and depressive disorders \cite{rice1998self,dori1999depression}. Self-esteem alone has no direct contribution to depression, but self-esteem interact with accumulate stress highly predicts depression \cite{miller1989self}.  Shin (Shin, 1993) found that when self-esteem, social support and cumulative stress are introduced to the regression analysis of depression, only self-esteem accounts for significant additional variance \cite{shin1993factors}. 

\subsubsection{gender and cultural differences}

A large amount of literature has found significant gender differences in depressive and anxiety disorders. Nevertheless, gender is an influential factor to help-seeking behavior. One one hand, women in genderal have higher self-disclosure level, thus they are more willing to seek help. On the other hand, gender stereotype such as women are prone to affective disorder and men are prone to substance use disorder also reinforce the stigma and affect the help seeking behaviors. Gender differences also exist in the comorbid rate of depression and anxiety \cite{afifi2007gender}. 

Cultural differences also influence the development of mental disorders. There are cultural bound syndromes, which are disorders that are more likely to occur in certain culture and context. For example, obesity and anorexia nervosa in the United States \cite{ritenbaugh1982obesity}, social withdrawal in Japan \cite{teo2010hikikomori}. Cultural stigma also restraint the help-seeking behaviours of people from a certain culture. A person live in a culture full of stigmatizing image of mental health may internalize these ideas and suffer from a diminish in self-esteem and confidence \cite{corrigan1998impact,holmes1998individual}. 

In summary, a lot of the predisposing factors, such as personality, self-disclosure and self-esteem, are dependent on family rearing style. It's nearly impossible to extract childhood information of a person from social media data, because social media only exploded in the recent decade. However, it's possible to extract information that reflects predisposing risk factors from social media data. As a result, my project will involve studying some of these disposing factors. Predisposing factors makes up the vulnerabilty of mental disorders, but predisposing factors alone are not directly link to mental disorders. Their interaction with precipitating factors - stress, are most predictive of mental health.


\subsection{precipitating factors and perpetuating factors}

Life stress has been acknowledged as an important factors in illness onset (citation). It has been found that major life events, such as divorce, death of a closed friend or relative, being fired from work and so on are precipitating factors on mental illnesses (citation). However, the correlations between stress and health outcome is confound by many factors, such as the predisposing factors in the previous sections. As a result, the correlations between stress and health are modest (r =  0.2 - 0.4)  The correlations are much stronger among those who are already ill. cite (Aldwin and Gilmer, 2004).

\subsubsection{measurement of stress}
Most of the literature analyze stress from two perspective: psychological and environmental perspectives. The psychological perspective focuses on the individual's appraisal of their ability to cope with stress, while the environmental perspective focuses on the life events that cause the stress cite ((McGrath, 1970)). 

However, measuring stress level with the type of stressor is far from accurate due to individual differences in assigning meanings to life events. Objective rating of life events, for instance, assuming that bereavement must be extremely stressful, does not account for the individual differences.  Wortman
and Silver (1989) found that one-third of the bereaved spouses do not appear to be distressed at any point up to one year after the spouse's death \cite{wortman1989myths}. Thus, subjective ratings of stress are more predictive to mental health conditions.  Scales that measure the subjective rating of stress include Schedule of Recent Life Experiences (SRE) cite(), the Social Readjustment Rating Scale (SRRS) and so on (cite). Another way to measure stress is the interview. Interview measurement is particularly critical in diagnosis because the timing of exposure to the stressor indicate the onset of the illness.


\subsubsection{types of stress}
\textbf{Random stressors and systematic stressors}. Random stressors occur with relatively equal probability across various social group, e.g. miscarriages. Systematic stressors happen in specific social location or social group, e.g being robbed. 

\textbf{Chronic and acute stressors.}  Acute stressors result in onset of illness, while chronic stressors expose an individual to long-term stress. Both of them play an important role in the development of mental disorders. However, sometimes the boundary of chronic and accute stressors are not clear cut. In fact, many negative events are not uniformily evaluated as discrete or chronic (Avison and Turner, 1988). For instance, a car accident is an acute stressor, but the injuries follow with the accident can take years to recover, which makes this accident a chronic stressor.

\textbf{Change events and negative change events.} There are consistent findings that supports a positive relationship between negative change event and mental illness. A large amount of studies shows that negative events rather than positive events precipitate symptoms} (Rabkin and Streuning, 1976; Turner and Wheaton,1995). Whereas, the findings for positive event and mental illness are contradictory. cite (Zautra and Reich (1983). As a result, life event checklists emphasis the undesirability as the most important factor to mental illness. 

\subsubsection{Other aspects of stress} 
\textbf{Uncontrollability of event.} It is also important to note that uncontrollable event is more stressful than those that are under an individual's control. For example, initiate a breakup with a partner is less stressful than being told the bad news by a partner; Being fired is more stressful than quitting a job. 


\textbf{culmulative of stress and role strain.} It is believed that the accumulation of stress is more stressful than an isolated life events (Evans, 2004). The literature suggests that the duration of stress effect usually lasts for 6 months (Depue and Monroe, 1986; Norris and Murrell, 1987). In addition, chronic role strain and hassles may also increase the severity of the illness. 
 
Chronic role strain is a systematic stressor, which emphasis the social causes of psychological distress. Intuitively, social role often brings chronic role strain, for example, the role of a working mother need to thrive in her career and take care of children at home. As a consequence, in the measurement of stress, respondent who take on more social roles should report higher tension in stressful events. However, social role is, in fact, a confunding facter, because the number of roles one taken is linked to one's social competence. Social competence is negatively correlated with distress. 


\subsection{protective factors}
\subsubsection{life experience}
The earliest literature of psychology has discovered the relationship with life expereince and psychological disorders. Later on, a considerable body of reseach find that life experience varied in their risks to develop mental disorders. Even for children who have experience the most severe  stressors, only half of them succumb to the adversities (Rutter, 1979). The risks of life experience is mediated by predisposing factors and protective factors. 

Adverse life experience itself can be a protective factor. There is large amount of animal studies show that physical stresses in early life lead to neurological changes that improve the animal's resistence to stress happen later in life. (Hennesy and Levine, 1979). In certain circumstances, adversity enhance an individual's psychological hardness. Whether one will develop psychological hardness depends on predisposing factors and timing of the event. Timing and meaning are important as timing affects the meaning attached to the event and meaning affects individuals appraisal of whether the event is postive or threatening (Rutterm, 1981). cite (Resilience in the Face of Adversity Protective Factors and Resistance to Psychiatric Disorder )

\subsubsection{cognition}
People's cognition affect their appraisal of their ability to cope with the adverse situation. On one hand, cognitive skills are linked to the development of self-esteem. Whether one is successful in the domain of importance is crucial to the development of self-esteem and this is largely dependent on cognitive skills. On the other hand, coping stragies differentiate people who think they can cope with the situation and those who can't. The characteristics of many people who have experience chronic stress is that they feel helpless and they can't do anything to change the situation cite (Resilience in the Face of Adversity Protective Factors and Resistance to Psychiatric Disorder ). 


\subsubsection{social support}

Social embeddedness, perceived support and enacted support are three major aspects of social support. 
Social embeddedness refers to an individual's connections to significant others and may include one's role in the family or the community and one's social network. Social embeddedness both create stress and provide support. On one hand, social roles can lead to role strains but reflect competence, which is an important component in self-esteem. On the other hand, tensions in social ties create stress, but healthy social ties brings social support. As a consequence, social embedness, which is reflected by the social network on social media could be a confounding factor in studying social support. 

Enacted social support refers to the support that other people provide. Perceived social support refers to whether an individual perceive that they have received the support, it focuses on the adequacy and availability of the support.  Perceived support is more likely to have stress-buffering effect because they influence the appraisal of stressful situations. (S. Cohen and McKay (1984), as well as by Cutrona and her colleagues (Cutrona), (Cutrona
and Russell, 1990). As a result, whether one perceive they receive support is more important to health than whether they actually receive support. 


\textbf{Types of social support.} The types of social support are slightly different in various theories. Gottlieb (1978) identified four classes of social support from 26  26 categories of helping behaviors. The four classes are emotionally sustaining behaviors, problem-solving behaviors, indirect personal influence, and environmental action. (House and Kahn, 1985 proposed three major types of support: emotional, informational, and tangible support. Cutrona, Suhr, and MacFarlane, 1990 proposed esteem, network (companionship), informational, tangible, and emotional support.


\subsection{Conclusion}
Summary, protective factors may be positive or negative life experience, or may not be an experience at all. It could be the quality of a person, such as the predisposing factors (self-esteem, personality, self-disclosure) or it could be social support. All these factors interact with stress in predicting the risk of mental illness. Hence, the study of stress's effect on mental health can not be isolated from the factors that entangles with it. 


\section{Psychological disorders and their symptoms}

\subsection{Diagnostic criteria}
The aim of my research is to study symptom related behaviours on social media but not for diagnosis. Hence, identifying symptoms from the diagnostic manual is one of the crucial steps. The Diagnostic and Statistical Manual of Mental Disorders, Fifth Edition (DSM-5) is the latest version of diagnostic criteria manual. DSM-5 diagnostic criteria contain 22 categories of disorders, some of the disorders has been widely studied in the social media context. There are computer classifiers that claim to be able to diagnose disorders automatically.

However, the diagnosis of mental disorders is extremely complicated. First of all, there are multiple subcategories under each major disorder categories. For example, depressive disorders include disruptive mood dysregulation disorder, major depressive disorder, persistent depressive disorder, premenstrual dysphoric disorder and so on. Diagnostic criteria between the subcategories are very similar, the distinction between the depressive disorder subcategories mainly lies in the onset time and how long the symptoms prolong. 

Second, a major concern of diagnosis is co-morbidity, which means an individual is diagnosed with more than one disorders at the same time. Kessler and others conduct a National Co-morbidity Survey and found that the 'pure' cases are very rare cite (Kessler et al., 1994). Schwalberg et al. (1992) found the comorbidity between eating disorders and anxiety is 70–80\%, Judd et al. (1998) found the co-morbidity between general anxiety disorder and depression is 80\%. The co-morbidity between bipolar disorder and another psychological disorder is 61\%. (Taman and Ozpoyraz (2002)). The comorbidity may refelct a common vulnerability. For instance, poor coping skills  (Andrews 1996) is one of the vulnerabilities that expose an individual to higher risks of anxiety and depressive disorders. 

Nevertheless, DSM-5 also stress culture and gender related diagnostic issues. In many cultures, somatic symptoms are predominant symptoms in depressive disorders. A majority of depressive disorder cases are unrecognized in primary care settings because various culture or gender related somatic symptoms are more predominant than affective symptoms. In addition, the female-to-male prevalence ratio, ages of onset various across cultures cite(Cognitive Behavioural Processes Across Psychological Disorders: A transdiagnostic approach to research and treatment). 


\subsection{Study psychological disorders with social media data}

Social media behaviours reflect some of the symptomatic behaviours, especially mood disorders. Hence, psychological disorders feature with affective symptoms are well studied with social media data. These disorders including Schizophrenia, bipolar, depression, and anxiety disorders cite(). Social media studies usually do not cover psychological disorders focus on physical symptoms or cognitive functions because these symptoms are less likely to be observed from social media data. For example, Obsessive-Compulsive disorder, sexual dysfunction and so on. 

Symptoms can be observed in social media data include affect and sleep patterns. Occassionally social media users mention the drug they are taking or specific physical symptoms they experienced on social media platforms. However, the disclosure of this kind of information is rather rare in social media platform such as Facebook and Twitter, it's more often on anonimous platform such as Reddit. I will discuss the differences between social media platform in the next section.

\textbf{Depressive disorders.} Some of the major symptoms included in the diagnostic criteria and occur in most of the subcatergories include: depressed mood, loss of interest or pleasure, insomnia or hypersonmnia, fatigue, anxiety, irritability and anger.


\textbf{Anxiety disorders.} The distinctions between different subcategories under anxiety disorders lies in the themes or context that trigger the symptoms. The symptoms include,  excessive fear or anxiety concerning the theme, repeated nightmares involving the theme, repeated complaint of physical symptoms (e.g. headache, stomachache). A common feature of anxiety disorders is that the fear, anxiety or avoidence is persistent for a long time.


\textbf{Bipolar disorder.} Bipolar disorders include a manic epispde and a depressive episode. During the manic episode, patient has an inflated self-esteem, thought racing, distractibility and decreased need for sleep.


Based on the complexity of diagnosis process, it is unlikely to have machine learning classifer that are able to predict a task that is already difficult in human decision.

\subsection{transdisgnostics framework}

transdisgnostics framework and its advantages, comorbidity


transdiagnotics framework is good for treatment but not for diagnositic because Accounting for the differences is likely to be a major challenge to a transdiagnostic perspective.




\section{Social media study on mental health}

types of social media platform and their differences

types of social media study, fits in the 4p framework






\section{Social media data analysis techniques}

low social status increases the risk of being exposed to a number of adverse conditions, both physical and psychological (Adler, Marmot, McEwen and Stewart, 1999; Adler and Matthews, 1994). 











