%!TEX root = ../thesis.tex
%*******************************************************************************
%****************************** Second Chapter *********************************
%*******************************************************************************

\chapter{Background}

introduction to 4P, 

Traits that are related to psychological disorders
psychological disorder symptoms

\section{Factors contributing to mental health}
% Uncomment this line, when you have siunitx package loaded.
%The SI Units for dynamic viscosity is \si{\newton\second\per\metre\squared}.
The 4P model is a biopsychosocial approach to analyze mental health development and progression \cite{}. The 4P model suggests that there are four types of factors contribute to development of all psychological disorders, each type of factors can be further divided into biological factors, psychological factors and social factors. The four types of factors are: predisposing risk factors, precipitating risk factors, perpetuating risk factors and protective factors. 

Predisposing risk factors make an individual more susceptible to a psychological disorder. For example, genetic component, gender differences, personality traits and parental modeling. Precipitating risk factors contribute to the onset of psychological disorders, for instance, traumatic event or stressful life events. Perpetuating factors maintain the symptoms span from psychological disorders, such as chronic stress, cognitive bias, social stigma, social roles. Finally, protective factors build up the resilience of an individual \cite{}. Figure \ref{tab:4P} demonstrates the relationships of these factors. Note that some of the precipitating factors can be perpetuating factors if they exist for a long time. 

We attempt to study some of the factors via the lenses of social media platform. Social media behaviors reflect psychological and social factors to some extend but it's nearly impossible to inspect biological factors from social media data, except for gender differences. Another factor that is difficult to inspect in social media at this stage is childhood experience. As a result, my PhD project will only focus on inspecting factors that are available in social media data.

% Please add the following required packages to your document preamble:
% \usepackage{graphicx}
\begin{table}
\caption{Biopsychosocial Approaches in 4P Model}
\label{tab:4P}
\resizebox{\textwidth}{!}{%
\begin{tabular}{|l|p{2cm}|p{4cm}|p{4cm}|}
\hline
\textbf{4P factor model}            & \multicolumn{3}{l|}{\textbf{Biopsychosocial approach}}      \\ \hline
                           & \textbf{biological factors} & \textbf{psychological factors}          & \textbf{social factors}                                      \\ \hline
\textbf{Predisposing risk factors}  & genetic, gender    & personality traits, mood stability, self-disclosure, self-esteem, perfectionism, sensitivity to stress       & parental modelling,                                 \\ \hline
\textbf{Precipitating risk factors} & physical injuries  & stress/trauma                                                                                                & change in family dynamics                           \\ \hline
\textbf{Perpetuating risk factors}  & chronic diseases   & chronic stress, malnutrition, crowding, pollution                                                            & social stigma, social roles, culture, social status \\ \hline
\textbf{Protective factors}         & genetic            & effective coping skills, high resilience, high cognition skills, high self-esteem, low sensitivity to stress & social support                                      \\ \hline
\end{tabular}%
}
\end{table}

\subsection{predisposing factors}
When individuals are exposed to a stressor, predisposing factors influence how people experience stress and how they cope with it. These factors affect an individual in the probability to encounter stressors. Note that people select their environment and shape them. They also affect people's tendency to react to situation and their tendency in coping. \cite{} (Suls & Martin, 2005; Vollrath, 2001; cf. Watson, David & Suls, 1999)

\subsubsection{personality}
There has been consistent evidence showing that personality plays an important role in the experience of stress, in particular, coping with stress (e.g., Kiecolt-Glaser, McGuire, Tobles & Glaser, 2002).

There are many types of personality models, such as, Myers–Briggs Type Indicator (MBTI),Cattell’s 16 Factor Model, Big Five Model, among which the five factor personality model (Big5) is most widely adopted in personality studies. Big5 suggests five board dimensions to describe human personality:  openness to experience, conscientiousness, extraversion, agreeableness, and neuroticism. 


\subsubsection{mood stability}
\subsubsection{self-disclosure}
\subsubsection{self-esteem}
\subsubsection{gender and cultural differences}


\subsection{precipitating factors and perpetuating factors}


low social status increases the risk of being exposed to a number of adverse conditions, both physical and psychological (Adler, Marmot, McEwen & Stewart, 1999; Adler & Matthews, 1994). 


\section{Types of psychological disorders and symptoms}

\section{Social media data and its retrieval}

\section{Analysis Techniques}