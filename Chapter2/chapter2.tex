%!TEX root = ../thesis.tex
%*******************************************************************************
%****************************** Second Chapter *********************************
%*******************************************************************************

\chapter{Background}

introduction to 4P, 

Traits that are related to psychological disorders
psychological disorder symptoms

\section{Factors contributing to mental health}
% Uncomment this line, when you have siunitx package loaded.
%The SI Units for dynamic viscosity is \si{\newton\second\per\metre\squared}.
The 4P model is a biopsychosocial approach to analyze mental health development and progression \cite{}. The 4P model suggests that there are four types of factors contribute to development of all psychological disorders, each type of factors can be further divided into biological factors, psychological factors and social factors. The four types of factors are: predisposing risk factors, precipitating risk factors, perpetuating risk factors and protective factors. 

Predisposing risk factors make an individual more susceptible to a psychological disorder. For example, genetic component, gender differences, personality traits and parental modeling. Precipitating risk factors contribute to the onset of psychological disorders, for instance, traumatic event or stressful life events. Perpetuating factors maintain the symptoms span from psychological disorders, such as chronic stress, cognitive bias, social stigma, social roles. Finally, protective factors build up the resilience of an individual \cite{}. Figure \ref{tab:4P} demonstrates the relationships of these factors. Note that some of the precipitating factors can be perpetuating factors if they exist for a long time. 

We attempt to study some of the factors via the lenses of social media platform. Social media behaviors reflect psychological and social factors to some extend but it's nearly impossible to inspect biological factors from social media data, except for gender differences. Another factor that is difficult to inspect in social media at this stage is childhood experience. As a result, my PhD project will only focus on inspecting factors that are available in social media data.

% Please add the following required packages to your document preamble:
% \usepackage{graphicx}


% section section_name (end)
\begin{table}
\caption{Biopsychosocial Approaches in 4P Model}
\label{tab:4P}
\resizebox{\textwidth}{!}{%
\begin{tabular}{|l|p{2cm}|p{4cm}|p{4cm}|}
\hline
\textbf{4P factor model}            & \multicolumn{3}{l|}{\textbf{Biopsychosocial approach}}      \\ \hline
                           & \textbf{biological factors} & \textbf{psychological factors}          & \textbf{social factors}                                      \\ \hline
\textbf{Predisposing risk factors}  & genetic, gender    & personality traits, mood stability, self-disclosure, self-esteem, perfectionism, sensitivity to stress       & parental modelling,                                 \\ \hline
\textbf{Precipitating risk factors} & physical injuries  & stress/trauma                                                                                                & change in family dynamics                           \\ \hline
\textbf{Perpetuating risk factors}  & chronic diseases   & chronic stress, malnutrition, crowding, pollution                                                            & social stigma, social roles, culture, social status \\ \hline
\textbf{Protective factors}         & genetic            & effective coping skills, high resilience, high cognition skills, high self-esteem, low sensitivity to stress & social support                                      \\ \hline
\end{tabular}%
}
\end{table}

\subsection{predisposing factors}
When individuals are exposed to a stressor, predisposing factors influence how people experience stress and how they cope with it. These factors affect an individual in the probability to encounter stressors. Note that people select their environment and shape them. They also affect people's tendency to react to situation and their tendency in coping. \cite{} (Suls and Martin, 2005; Vollrath, 2001; cf. Watson, David and  Suls, 1999)

\subsubsection{personality}

Consistent evidence has been found that some dimensions in personality co-occurred with certain mental disorders, but there's no direct link between personality and mental health problems. Personality is largely shaped by child-rearing practices and family interactions (cite). As a result, the link between personality and mental health can be traced back to family rearing style.

There are many types of personality models, such as, Myers–Briggs Type Indicator (MBTI),Cattell’s 16 Factor Model, Big Five Model, among which the five factor personality model (Big5) is most widely adopted in personality studies. Big5 suggests five board dimensions to describe human personality:  openness to experience, conscientiousness, extraversion, agreeableness, and neuroticism. There has been consistent evidence showing that personality plays an important role in the experience of stress. Individuals with high neuroticism often experience more negative emotions (cite), and negative emotions affect how an individual cope with stress \cite{kiecolt2002emotions}. Neurotism has been found to be an increase risk factor for depression and anxiety. \cite{talley1986association,cattell1961meaning} Internet addiction often occurs in introverted people \cite{xiuqin2010mental}. However, personality and mental health do not have a direct causation relationship. There are many mediating factors between personality types and mental disorders. For example, rumination and worrying mediate neuroticism and depression \cite{roelofs2008rumination};

\subsubsection{self-disclosure}
Self-disclosure is the amount of information an individual disclose to significant others. There are three major parameters of self-disclosure: the amount of information disclosed; depth of the information disclosed and the time spent on describing the information. The evaluation of self-disclosure must involve the three parameters cite(self-disclosure, a literature review).

Self-disclosure is important to mental health problem because it affects one's help seeking behaviors. Individuals with high self-disclosure level are more likely to seek professional help (Hinson and Swanson, 1993). Self-disclosure is also a critical problem for psychotherapist. On one hand, patients often go to psychotherapy to seek solution to loneliness, which is a result of difficulty in building intimate relationships. One the other hand, patients wish to conceal themselves. (The Shared Experience and Self-Disclosure Martin Fisher, Self-Disclosure in the Therapeutic Relationship). 

Self-disclosure is, again, shaped by the family rearing style. Fisher(1982) proposes that when infant/child involves in a communication with significant others, they receive positive or negative feedback. The feedback provided form a feedback loop. The positive feedback loop encourage a child to disclose more, whereas, the negative feedback loop does the opposite. Failing to disclose results in an accumulation of secrets, the greater the degree of secrets, the more alienation of self and others. cite() Besides family rearing style, self-disclosure is also affected by gender, culture, race and the relationship with the significant others.

\subsubsection{self-esteem}
Self-esteem is the beliefs and evaluation individuals hold about themselves (Burns, 1982). These beliefs are inner influences that guild and government individuals through out their lives. The development of self-esteem is dependent on support and approval from significant others and self-perceived competence in the domains that are important to self. The distance between ideal self and real self is critical to the development of self-esteem cite. (Self-esteem in a broad-spectrum approach for mental health promotion)

Self-esteem is a protective factor in mental health because self-esteem and the ability to face challenge determine what happens to an individual in challenging situations. Conversely. poor self-esteem plays a critical role in the development of mental disorders. cit (Self-esteem in a broad-spectrum approach for mental health promotion)   High self-esteem is related to better life satisfaction (Zimmerman, 2000).  Self-esteem is a strong predictor of subjective well-being (Furnham and Cheng, 2000). 

Whereas, Low self-esteem lead to maladjustment  (Garmezy, 1984), depressed mood and depressive disorders (Rice et al.,1998; Dori and Overholser, 1999). Self-esteem alone has no direct contribution to depression, but self-esteem interact with accumulate stress highly predicts depression (Miller et al., 1989).  Shin (Shin, 1993) found that when self-esteem, social support and cumulative stress are introduced to the regression analysis of depression, only self-esteem accounts for significant additional variance (cite). 

\subsubsection{gender and cultural differences}

summary, a lot of these factors are mediating factors between family rearing style and mental illness. We don't know the family rearing style from social media, but we can infer these factors from social media behaviors. 

\subsection{precipitating factors and perpetuating factors}


low social status increases the risk of being exposed to a number of adverse conditions, both physical and psychological (Adler, Marmot, McEwen and Stewart, 1999; Adler and Matthews, 1994). 



\section{Types of psychological disorders and symptoms}

\section{Social media data and its retrieval}

\section{Analysis Techniques}