%!TEX root = ../thesis.tex
%*******************************************************************************
%****************************** Second Chapter *********************************
%*******************************************************************************

\chapter{Background}

In this chapter, we introduce some of the psychological constructs that affect the development of mental disorders. The contribution of these constructs to mental disorders is explained by the underlying psychological mechanisms of mental disorder. These mechanisms are described in the case formulation process of clinical diagnosis, which is a crucial procedure for diagnosing mental disorders.  We also discuss which constructs have been covered in social media studies and what are the research gaps and limitations in these studies.

\section{Case formulation models}
% Uncomment this line, when you have siunitx package loaded.
%The SI Units for dynamic viscosity is \si{\newton\second\per\metre\squared}.
This section explains the components in case formulation. The standard procedures for diagnosing psychological disorders must go through: assessment, case formulation and treatment planning. Case formulation is used to find out the mechanisms that cause and maintain a psychological disorder. A case formulation can answer questions such as, what are the problems? What cause the problems? What maintain the problems? Understanding the mechanisms helps us to identify what factors or psychological components are important for the development of psychological disorders. 

Two case formulation models are widely adopted in the literature: 1) the biopsychosocial model \cite{george1980clinical}, which refers to biological factors, psychological factors and social factors.  2) the five P's of Case formulation \cite{macneil2012diagnosis}. The 5P model consists of three risk factors (predisposing factors, precipitating factors, perpetuating factors), presenting factors and protective factors. The risk factors and protective factors can be combined with the biopsychosocial model (see Table \ref{tab:4P}). 

Predisposing factors cause an individual to be more susceptible to a psychological disorder, for example, genetic component, gender differences, personality traits and parental modeling. Precipitating factors contribute to the onset of psychological disorders. A typical example for precipitating factor is stressful life events. Perpetuating factors maintain the symptoms span from psychological disorders. They include chronic stress, cognitive bias, social stigma, social roles and so on. Finally, protective factors build up the resilience of an individual \cite{macneil2012diagnosis}. Figure \ref{tab:4P} demonstrates these four factors combine with the biopsychosocial model.  

%In my PhD projects, we attempt to study some of the factors via the lenses of social media platforms. Social media behaviors reflect some of the factors in the case formulation models, especially the psychological and social factors.  However, the limitation of using social media data to inspect risks factors is that biological and adverse childhood experience are usually not available in the data. As a result, we will only focus on factors that are reflected in social media data.


% section section_name (end)
\begin{table}
\caption{Case formulation models}
\label{tab:4P}
\resizebox{\textwidth}{!}{%
\begin{tabular}{|l|p{2cm}|p{4cm}|p{4cm}|}
\hline
\textbf{5P factor model}            & \multicolumn{3}{l|}{\textbf{Biopsychosocial approach}}      \\ \hline
                           & \textbf{biological factors} & \textbf{psychological factors}          & \textbf{social factors}                                      \\ \hline
\textbf{Predisposing risk factors}  & genetic, gender    & personality traits, mood stability, self-disclosure, self-esteem, perfectionism, sensitivity to stress       & parental modelling,                                 \\ \hline
\textbf{Precipitating risk factors} & physical injuries  & stress/trauma                                                                                                & change in family dynamics                           \\ \hline
\textbf{Perpetuating risk factors}  & chronic diseases   & chronic stress, malnutrition, crowding, pollution                                                            & social stigma, social roles, culture, social status \\ \hline
\textbf{Protective factors}         & genetic            & effective coping skills, high resilience, high cognition skills, high self-esteem, low sensitivity to stress & social support                                      \\ \hline
\end{tabular}%
}
\end{table}

\section{Predisposing factors}
Researchers have found that there are some factors that cause an individual to be more susceptible to mental disorders. They group these factors as predisposing risk factors for mental disorders. The interaction between stress and predisposing factors influences how people experience stress and how they cope with it. Predisposing factors affect an individual's probability to encounter stressful events because people select and shape their environment. Nevertheless, they also affect people's tendency to react to situation and their ability in coping \cite{suls2005daily,vollrath2001personality,watson1999personality}. In this section, we will introduce some predisposing factors that have been studied in the context of social media. 

\subsection{Personality}
 The categories of personality vary in different theoretical models. There are many types of personality models, such as, Myers–Briggs Type Indicator (MBTI), Cattell’s 16 Factor Model, Big Five Model, among which the five factor personality model (Big5) is most widely adopted in personality studies. Big5 suggests five broad dimensions to describe human personality: openness to experience, conscientiousness, extroversion, agreeableness, and neuroticism. 

Personality is largely shaped by child-rearing practices and family interactions \cite{kluft1985childhood}.  There has been consistent evidence showing that personality plays an important role in the experience of stress, especially neuroticism. For instance, neuroticism has been found to be a risk factor for depression and anxiety \cite{talley1986association,cattell1961meaning}. Individuals with high neuroticism often experience more negative emotions \cite{larsen1991personality}, thus affecting how an individual cope with stress \cite{kiecolt2002emotions}.  In addition to that, there are many factors mediating the relationships between personality types and mental disorders. For example, rumination and worrying mediate neuroticism and depression \cite{roelofs2008rumination};

\subsubsection{social media studies on personality}
Personality in the Big-5 model has been widely studied in social media context  \cite{ross_personality_2009,ma_anonymity_2016,hollenbaugh_facebook_2014,rife_participant_2016,farnadi_computational_2016,ross_personality_2009,moore_influence_2012,back_facebook_2010,seidman_self-presentation_2013,hughes_tale_2012,hong_analysis_2014,skues_effects_2012,zhang_gratifications_2011,lee_personality_2014,jenkins-guarnieri_relationships_2012,blachnio_psychological_2013,wang2013share}. These studies mainly cover two issues: 1) How does personality influence the choice of social media platforms. For example, personality influences whether a person has a preference to use Facebook or Twitter \cite{hughes_tale_2012}. People with high conscientiousness tend to use Twitter, whereas, sociable people tend to use Facebook \cite{ross_personality_2009}. 2)Social media behaviors reflect personality. Extroverts are more active in Facebook activities and people with high neuroticism are less likely to write comments \cite{skues_effects_2012}. Individuals high in openness are more likely to post and share intellectual topics \cite{marshall_big_2015}. Some researchers adopted other personality models instead of Big5 their studies. They found that people who are high in Narcissism use Facebook for attention seeking. They are also more likely to update their diet and exercise routine \cite{marshall_big_2015,andreassen_relationship_2017}. 

\subsection{Self-disclosure}
Self-disclosure refers to the information individuals disclose to their significant others. There are three major parameters of self-disclosure: the amount of information disclosed; depth of the information disclosed and the time spent on describing the information. The evaluation of self-disclosure must involve the three parameters \cite{cozby1973self}.

Self-disclosure is shaped by the family rearing style, gender, culture, race and the relationship with the significant others. Fisher proposed that self-disclosure was shaped by the feedback loop provided by the care takers. When an infant/child involves in a communication with significant others, they receive positive or negative feedback \cite{patterson2002recent}. The feedback provided by the care takers form a feedback loop. The positive feedback loop encourages a child to disclose more, whereas, the negative feedback loop does the opposite. Failing to disclose results in an accumulation of secrets, the greater the degree of secrets, the more alienation of self and others. \cite{patterson2002recent}. 

\subsubsection{Social media studies on self-disclosure}
Self-disclosure is an important psychological construct in social media studies. On the one hand, it predicts how much information an individual is willing to disclose on social media platforms. Studies have shown that content intimacy not only regulates self-disclosure in face-to-face communication, but also in the social media settings \cite{ma_anonymity_2016}. The level of exposure of information about relationship issues, address, phone numbers, email address and health history were much lower than in the physical world. Whereas, photos about the user and the user's family were more likely to be disclosed on Facebook than in the physical world. The exposure level of positive and negative emotions, places they visit, political views, relationship status, work and study places were similar on Facebook and in the physical world. The information they disclosed in these domains is highly consistent with real life information \cite{villela2015contrasting}. In addition, age and loneliness also predict Facebook disclosure \cite{malik2016uses}. 


On the other hand, Self-disclosure affects one's help seeking behaviors. Patients often go to psychotherapy to seek solution to loneliness, which is the result of difficulty in building intimate relationships. However, patients also wish to conceal themselves \cite{fisher1990shared, stricker1990self}.  Individuals with high self-disclosure level are more likely to seek help and open up themselves in therapy sessions \cite{hinson1993willingness}. Anonymous social media platforms enable users to have higher disclosure level \cite{andalibi_announcing_2018}. Therefore, social media studies about disclosing a sensitive event and support seeking behaviour often use data from anonymous social media sites (e.g. Reddit). It's necessary to consider the disclosure level of a user while evaluating the accuracy of the machine learning psychological component because a lot of information might be missing from people with low self-disclosure level.


\textbf{Amount, depth and breath}. The social media studies cover the three dimensions in self-disclosure: amount, depth and breath. Hollenbaugh and colleagues studied how social media user's behaviours reflect the three dimensions of self-disclosure. They found that those who posted a large amount of information about themselves mainly use Facebook to maintain existing relationships. People who were low in conscientiousness, agreeableness and social cohesion tend to disclose more information. Individuals who were high in extroversion, less conscientious but more agreeable tend to provide greater depth in their disclosure. Those with lower self-esteem and less neuroticism disclosed information that covers more varieties of topics \cite{hollenbaugh_facebook_2014}.

\textbf{Choice of social media sites}. Self-disclosure also influences the choice of social media platforms. Studies have shown that motivation to communication predicts the preference for using Facebook while other motivations may be influential in the decision of which social media platforms to use \cite{jenkins2012relationships}. Individuals with low self-disclosure level might not often use Facebook or Twitter, because they are often tied to a user's real-life social network. However, they might often use anonymous platform such as reddit.

\textbf{Disclosure of stressful life events}. 
Whether one discloses stressful experiences on social media depends on the evaluation of the feedback they will receive. Users weight the benefits (provide social support) and drawbacks (prolong emotional pain) before they make any sensitive disclosure.  Gaining public support is the major motivation to disclose sensitive information \cite{vitak2017benefits}. 

There are also studies that focus on how important or stressful life events are disclosed on Facebook. It has been shown that people tend to share positive life events indirectly and negative life events directly \cite{bevan_how_2015}. 

\subsection{Gender and Cultural differences}
 Cultural differences should not be ignored if we use social media data to infer mental disorder symptoms because patients from different cultures could have cultural bound syndromes, which are symptoms that occur more often in certain culture and context. For example, obesity and anorexia nervosa in the United States \cite{ritenbaugh1982obesity}, or social withdrawal in Japan \cite{teo2010hikikomori}. In addition, cultural stigma restraints the help-seeking behaviours of people from certain cultures. People who live in a culture full of stigmatizing image of mental health may internalize these ideas and suffer from a diminish in self-esteem and confidence \cite{corrigan1998impact,holmes1998individual}. Help seeking behaviours from those who come from a highly stigmatized culture and a more accepting culture could be very different.

In addition, there are gender differences in the expression and disclosure of mental disorder symptoms. A lot of studies have shown significant gender differences in depressive and anxiety disorders \cite{afifi2007gender}. Nevertheless, gender is an influential factor for help-seeking behaviors. Women in general have higher self-disclosure level, thus they are more willing to seek help. 

Cultural differences in the social media context haven't been widely explored yet. Some studies show that the disclosure of traumatic events on social media varies across cultures \cite{freitag2011talking,de2016quantifying}. De Choudhury and colleagues studied the disclosure of mental illness on social media sites and found significant differences between content shared by male and female users and users from different countries. Male users showed more negativity and they are more reluctant to seek social support.  Park and others, found that female Facebook users were warmer, more compassionate and slightly more assertive in their language \cite{park_women_2016}. In addition, people from various cultures display different language patterns when disclosing mental health problems \cite{de_choudhury_gender_2017}. Pennebaker et al., found that depressed individuals who wrote in Spanish showed more relational concern, whereas, depressed individuals who wrote in English showed are more concern about medication \cite{ramirez2008psychology}.

\section{Precipitating factors and perpetuating factors}
This section discusses stress as the precipitating factor and perpetuating factor of psychological disorders. Stress has been acknowledged as an important factor for the onset of psychological disorders \cite{langner1963life}. Since people often disclose their stressful experience on social media platforms, social media platform provides us with a good source to study stress and how stress interact with the risk factors of an individual. In order to know how we can utilize social media data to study stress, we introduce the measurement of stress, type of stress and the other perspectives of stress.

\subsection{Measurement of stress}
Most of the literature analyzes stress from either psychological or environmental perspectives. The psychological perspective focuses on individuals' appraisal of their abilities to cope with stress, while the environmental perspective focuses on life events that cause the stress \cite{mcgrath1970social}. The assessment of stress is based on the two perspectives.

Intuitively, people tend to think the magnitude of stress is dependent on the event that invokes the stress. For example, the death of a friend is more stressful than being fired from work. As a result, the type of stressor reflects the stress level one experience. However, there are individual differences in making the meanings out of life events. What is considered to be very stressful for one person does not necessarily considered to be stressful for another person. Objective rating of life events, for instance, assuming that bereavement must be extremely stressful, does not account for the individual differences.  Wortman and Silver found that one-third of the bereaved spouses did not appear to be distressed at any point up to one year after the spouse's death \cite{wortman1989myths}. 

Whereas, subjective ratings of stress account for the psychological perspectives of an individual. It has been found to be more predictive to mental health conditions. Scales that are widely adopted to measure subjective rating of stress include Schedule of Recent Life Experiences (SRE) \cite{nuckolls1972psychosocial} and the Social Readjustment Rating Scale (SRRS). Note that another way to measure stress is interview. Interview measurement is critical in diagnosis because the timing of exposure to the stressor indicates the onset of the illness.

\subsection{Types of stress}
Stressors can be categorized according to whether they are random, whether they are chronic or acute or whether they are negative or positive. 

\textbf{Random stressors and systematic stressors}. Random stressors occur with relatively equal probability across various social groups, for instance, miscarriages. Systematic stressors happen in specific social location or social group, for example, being robbed. 

\textbf{Chronic and acute stressors.}  Acute stress refers to sudden stressful events. Negative life events often bring people acute stress, which might trigger the onset of illness. Since social media platform is a good source to study life events, acute stress has been explored in social media studies, for example, miscarriage \cite{andalibi_announcing_2018} and bereavement \cite{brubaker2012grief}.

On the other hand, chronic stress refers to stress that exists for a long period of time, for example, role stress, poverty and environmental stressors. Chronic stress maintains the symptoms and behaviours, yet it hasn't been well studied in the social media context.

Both acute and chronic stress play an important role in the development of mental disorders. However, the boundaries between chronic and acute stressors are not always clear cut. In fact, many negative events are not uniformly evaluated as discrete or chronic \cite{avison1988stressful}. For instance, a car accident is an acute stressor, but the injuries follow with the accident can take years to recover, which make this accident a chronic stressor.

\textbf{Change events and negative change events.} There are consistent findings that support a positive relationship between negative change events and mental illness. A large amount of studies show that negative events rather than positive events precipitate symptoms \cite{rabkin1976life,turner1995epidemiology}. Whereas, the findings for positive events and mental illness are contradictory \cite{zautra1983life}. As a result, life event checklists emphasize the undesirability as the most important factor to mental illness. 

Stress is often reflected in the valence of social media text, and a lot of social media studies focus on the detection of valence. Negative valence reflects mental disorder symptoms due to its association with negative changes. However, positive valence is often neglected in social media studies. Too much positive valence is often seen in people with a manic episode. 

\subsection{Other aspects of stress} 
\textbf{Uncontrollability of event.} It is also important to note that uncontrollable events are more stressful than those that can be controlled. For example, initiating a breakup with a partner is less stressful than being told the bad news by a partner; Being fired is more stressful than quitting a job. 

\textbf{Cumulative of stress and role strain.} It is believed that the accumulation of stress is more stressful than an isolated life events \cite{evans2013childhood}. For a single isolated event, literature suggests that the duration of stress effect usually lasts for 6 months  \cite{depue1986conceptualization,norris1987transitory}. 

In addition, chronic role strain and hassles may also increase the severity of the illness. Chronic role strain is a systematic stressor that emphasize the social causes of psychological distress. Social role often brings chronic role strain. For example, working mother is often required to both be successful in her career and take care of children at home. The role of being a mother and a career woman adds up the stress. However, social role is, in fact, a confounding factor, because the number of roles one taken is linked to one's social competence \cite{cohen1997measuring}, which has been found to be negatively correlated with distress.

\subsubsection{Social media studies on stress}
Social media studies on stress mainly include machine learning stress prediction models and machine learning life events. While studying life events using social media data, it is important to note that there are some limitations. Users report life events selectively and many users do not report negative life events on social media due to the concerns of self-presentation.

\textbf{Monitor Stress}. Stress is the major precipitating factor for mental illness.  Stress level is often reflected by the language and emoticons in the posts. Individuals with higher level of depression or anxiety level tend to post negative emotions on Facebook more frequently. The use of emoticons with positive emotions reflects stress level. This phenomenon is more prominent among the younger age group \cite{settanni2015sharing}.

A lot of studies use machine learning to predict stress levels based on social media data. Building stress prediction classifier often requires manually labeling posts that shows signs of stress according to the valence of the posts \cite{thelwall2017tensistrength,mogadala2012twitter,lin2014user}. 
Some of the studies link the stress prediction classifier to an app that provides therapeutic advice to users who show signs of stress. The aim of providing the advice is to improve users' coping skills \cite{li2014helping}. However, the therapeutic effect of these apps is not fully studied yet.

\textbf{Life events}. Life events are difficult to be identified from the text because users often report them indirectly, especially for negative events \cite{bevan2015important}. Hence some of the life events need to be be inferred from the text \cite{khodabakhsh2018detecting,abe2018predicting}.  Khodabakhsh and colleagues used language model to predict prominent life events. They found that if a sequence of tweets of the same topic occurred, these tweets were likely to report an important life event \cite{khodabakhsh2018detecting}. Dickinson and colleagues found that if users suddenly commented on topics that they were not usually interested in, they were likely to experience a prominent life event \cite{dickinson2015identifying}.

There are also studies that focus on stressful life events only. Stressful life events reflect stress from an objective perspective. Researchers are interested in how different stressful life events are disclosed on social media. For example, Haimson et al., have found that people engaged in breakup disclosure in two ways: one-to-many announcement on the wall of the social media platform or change private status. The way one reveals a relationship breakup has been found to be related to self-presentation \cite{haimson2018relationship}. Researchers also found those who experienced bereavement tend to use more first person singular pronouns, past tense verbs, adverbs, prepositions, conjunctions, and negations \cite{brubaker2012grief}. There is also cultural diversity in sharing traumatic events \cite{freitag2011talking}. 

%\subsection{Conclusion}
%Most of the social media studies on stress focus on analyzing people's reaction on acute stress. These studies often detect life events or inspect the behavior changes after a significant life events occurred. Whereas, chronic stress as the perpetuating factor hasn't been well studied on social media context. In addition, stress is often investigated as an isolated factor in social media studies. Risk factors and protective factors are intertwined with stress but their interactions with stress has not yet been studied in the social media context. Furthermore, positive valence is often ignored in social media studies that infer mental disorder symptoms.

\section{Protective factors}
Stress alone does not necessarily lead to mental health problems. Stress is detrimental to people who are susceptible to mental disorders and those who already developed mental disorders. The correlations between stress and health are modest (r =  0.2 - 0.4), but the correlations are much stronger among those who are already ill \cite{aldwin2004interface}. Hence, the interaction between stress and risk factors, protective factors are critical to the development of mental disorders. In this section, we introduce a several protective factors that have been covered by social media studies.

\subsection{Self-esteem.}
High self-esteem has a protective effect on mental illness. Studies have shown that the frequency of using social media site is indirectly linked to self-esteem and well-being. Using social media sites frequently enhances the number of relationship formed on the site and the feedback received. Adolescents with positive feedback have an increased self-esteem, and vice versa \cite{valkenburg2006friend}. Interestingly, people with low self-esteem are more likely to update romantic partner \cite{marshall_big_2015}.

Self-esteem is the beliefs and evaluation individuals hold about themselves. They guide and govern individuals through out their lives. The development of self-esteem is dependent on support and approval from significant others and self-perceived competence in the domains that are important to the self. The distance between the ideal self and the real self is critical to the development of self-esteem \cite{burns1982self}. 

Self-esteem is a protective factor in mental health because self-esteem and the ability to face challenge determine what happens to an individual in challenging situations. Conversely. poor self-esteem plays a critical role in the development of mental disorders. \cite{mann2004self}. High self-esteem is related to better life satisfaction \cite{zimmerman2000self}.  Self-esteem is a strong predictor of subjective well-being \cite{furnham2000perceived}. 

However, low self-esteem leads to maladjustment \cite{garmezy1984study}, depressed mood and depressive disorders \cite{rice1998self,dori1999depression}. Self-esteem alone has no direct contribution to depression, but self-esteem interacts with accumulated stress and highly indicative of depression \cite{miller1989self}.  Shin (Shin, 1993) found that when self-esteem, social support and cumulative stress are introduced to the regression analysis of depression, only self-esteem accounts for significant additional variance \cite{shin1993factors}. 

\subsection{Life experience}
 A considerable body of studies have shown that life experiences influence the risks of developing mental disorders. Adverse childhood experiences have a strong relationship with the development of mental disorders \cite{foege1998adverse}. However, even for children who have experienced the most severe stressors, only half of them succumb to the adversities \cite{rutter1979protective}, because the risk of adverse life experience is mediated by predisposing factors and protective factors. 

Adverse life experience itself can be a protective factor. There is large amount of animal studies show that physical stresses in early life lead to neurological changes that improve the animal's resistance to stress happen later in life \cite{hennessy1979stress}. In certain circumstances, adversity enhances an individual's psychological hardness. However, whether one will develop psychological hardness depends on many factors, such as the timing of the event or the meaning an individual assigns to the event. Meaning determines individuals' appraisal of whether the event is positive or threatening \cite{rutter1985resilience}. Timing is important because it affects the meaning attached to the event. One of the most important functions of counseling is to help individuals in their meaning making process. As a result, meaning making process should be considered when we study stressful life events in social media context. Whereas, most of the social media studies haven't included the meaning that an individual assigns to the stressful life events.

\subsection{Cognition}
Cognition affects people's appraisal of their abilities to cope with the adverse situation. People with better coping strategies tend to think they can cope with the situation. Many people who have experienced chronic stress feel helpless and think they can't do anything to change the situation \cite{rutter1985resilience}. Social media studies seldom cover cognition. In chapter 4, we discuss identifying cognitive distortion in Facebook posts.

\subsection{Social support}
Social support has been well studied in the social media context, especially on anonymous social media platforms, where users tend to have higher self-disclosure levels. The types of social support are slightly different in various theories. Gottlieb identified four classes of social support from 26 categories of helping behaviors \cite{gottlieb1978development}. The four classes are emotionally sustaining behaviors, problem-solving behaviors, indirect personal influence, and environmental action. House and colleagues proposed three major types of support: emotional, informational, and tangible support \cite{house1985social}. Cutrona and colleagues proposed esteem, network (companionship), informational, tangible, and emotional support \cite{cutrona1990interpersonal}.

\textbf{Aspects of social support.} Social embeddedness, perceived support and enacted support are three major aspects of social support. Social embeddedness refers to an individual's connections to significant others. It can refer to one's role in the family or one's social network. Social embeddedness both creates stress and provides support. On the one hand, social roles can lead to role strains but increase role competence. Role competence enhances self-esteem. On the other hand, tensions in social ties create stress, but healthy social ties bring social support. As a consequence, social embeddedness, which is reflected by the social network on social media platforms could be a confounding factor in studying social support. The number of connections doesn't necessarily predict the amount of support one will gain. When studying social embeddedness and social support, we study the number of social connections only, the tie strength is critical in predicting the amount social support provided.

Enacted social support refers to the support provided. Perceived social support refers to whether individuals perceive that they have received the support. Perceived social support focuses on the adequacy and availability of the support, hence it is more likely to have stress-buffering effect \cite{cohen1984social,cutrona1990type}. As a result, whether one perceives that the support they receive is more important to health than whether they actually receive support. However, most of the social media studies on social support look at the enacted support only. For example, the number of comments received and the linguistics characteristics of the comments. How the help seekers think about the comments are, however, often overlooked.

\subsubsection{Social media studies on social support}
\textbf{Social support provided on social media.} A large amount of social media studies on mental health focus on the social support provided on various social media platforms: Facebook and Twitter\cite{settanni2015sharing}, Reddit \cite{Sharma2018support} and even Instagram \cite{andalibi2017sensitive}. Studies also show that people use Instagram to share difficult experiences. Personal narratives, food and beverage, references to illness, and self-appearance concerns receive more social support. However, there's little aggression or support towards the harmful or pro-disease behaviours \cite{oh2015motivations}

Gaining social support is not the only reason that drives people to disclose their difficult life experiences on social media. Narration of a difficult life experience has been shown to have therapeutic effect \cite{dwivedi2006therapeutic}. However, a lot of people do not want to share their difficult experiences with people they know due to social stigma or concern of self-presentation. Anonymous social media platforms enable people to share their experience with strangers who don't know their social profile. Anonymous disclosure greatly enhances self-disclosure level. Anonymous disclosures on other sites facilitate disclosure on sites that link to real life profile (e.g. Facebook) \cite{andalibi2018announcing}. Anonymous disclosure on social media has been shown to have therapeutic effect on individuals who experience crisis \cite{saha2018social}. 

However, it has been shown that emotional competence and linguistic accommodation influence the therapeutic effect of disclosing difficult life experiences on social media. They affect one's expression and reception of emotional support. Giving and receiving emotional support in supporting groups has positive affect on people with high emotional communication competence but detrimental to those with low emotional communication competence \cite{yoo2014giving}. In addition to that, linguistic alignment with the online communities influences the amount of social support received. Communities that cater to mental health problems or have sensitive topics develop their own linguistic conventions. Those who maintain linguistic alignment with the community receive more social support \cite{Sharma2018support}. However, there's a major disadvantage in these communities. Those with low linguistic alignment might show greater vulnerability in their language, but nevertheless they receive less help. As a result, future design and studies on social media should not overlook those that receive less support on social media.

\textbf{Types of social support.} Multiple studies have identified the type of social support available on social media platforms \cite{yoo2014giving}. De Choudhury et al.,
\citet{Sharma2018support} identified the linguistic tokens that infer the type of social support on reddit. Emotional support usually includes words or phrases that indicate assent, e.g. 'I agree'. Esteem support often uses language that boosts one's moral or encourages hope \cite{Sharma2018support}. Informational support provides advice or suggestion. Sometimes audiences also provide instrumental support or network support \cite{de2017language}

 De Choudhury and colleagues also found that all communities on reddit offer both informational and emotional support, but these communities cater to different social support goals. Some communities requested and provided more emotional support, such as Mood Disorder communities. However, some communities requested and provided more informational support, such as Compulsive Disorder communities \cite{Sharma2018support}. This finding suggests that social support provided on social media matches with the needs of the support seeking individuals, thus it should have buffering effects against crisis.

\textbf{Social media site as a surveillance tool.} Some researchers attempt to utilize social network sites as surveillance tools \cite{reece2017forecasting}. For example, monitoring the behavioral change of new mothers \cite{de_choudhury_major_2013} or women who had experience miscarriage \cite{andalibi_social_2016}; Suicidal watch \cite{de2017language, de2016discovering} Monitoring psychological disorders \cite{andalibi_sensitive_2017,de2013predicting}.

However, we need to raise the awareness of risks in using algorithms in response to emotional pain. People share distress or emotions due to their sense of trust in the relationship they have or desire to have with users on social media sites \cite{brownlie2018looking}. Studies that engage in digital outreach should not impose any harmful effect on these relationships.

\textbf{Social embeddedness}
Social embeddedness refers to the connections that an individual has with their significant others. There are two aspects in the measurement of social embeddedness: social ties and social network \cite{sarason1983assessing}. Social network refers to number of the connections with others, whereas, social ties refers to the strength of connections. Some of the social media platforms provide a good source for the study of social network and social ties (e.g. Facebook). When researchers study social embeddedness, they need to bear in mind that the number of connections doesn't reflect the amount of social support available or received, the strength of social connections determine the amount of social support available and the quality of the support \cite{burke2010social}. Studies have shown that individuals who were high in loneliness tend to have more Facebook friends \cite{skues_effects_2012}.  

A few studies measure the tie strength on social media \cite{xiang2010modeling}. Gilbert and colleagues studied seven dimensions of tie strengths on Facebook: intensity, intimacy, duration, reciprocal services,structural,emotional support and social distance. They identified Facebook functionalities that were relevant to each dimension. For instance, they use wall words exchanged and participant-initiated wall posts to infer intensity. Using participant’s number of friends, days since last communication to infer intimacy variables \cite{gilbert2009predicting}. 

%\subsection{Conclusion}
%Summarily, among all the protective factors, social support is the most widely studied topic in social media context. However, most of these studies focused on enacted support, perceived support are overlooked. Other 
%other components in protective factors, such as self-esteem, meaning making process and coping skills are not studied yet. In addition, social media studies that mentioned social network often focus on the size of the network but overlooked the strength of ties.

\section{Presenting problems}
Presenting problems refer to the signs or symptoms of the psychological disorders in the assessment. In this section, we introduce the structure and the important issues discussed in the diagnostic manual. We also introduce some of the symptoms in various types of psychological disorders that have been studied in the social media context. Finally, we discuss what symptoms can be inferred from social media data and what have been overlooked in the current studies.

\subsection{Diagnostic criteria}
The Diagnostic and Statistical Manual of Mental Disorders, Fifth Edition (DSM-5) is the latest version of diagnostic manual \cite{american2013diagnostic}. The diagnostics manual describes 22 categories of disorders, each category consisting of several disorders. For example, depressive disorders include disruptive mood dysregulation disorder, major depressive disorder, persistent depressive disorder, premenstrual dysphoric disorder and so on. Diagnostic criteria between disorders in the same categories are very similar. 

Comorbidity is a major concern in diagnosis. Comorbidity means an individual is diagnosed with more than one disorder at the same time. Studies showed high comorbidity rate across many disorders. \citet{kessler1994lifetime} conducted a national comorbidity survey and found that the 'pure' cases were very rare \cite{kessler1994lifetime}. It has been shown that the comorbidity between eating disorders and anxiety reached 70–80\% \cite{schwalberg1992comparison}, anxiety disorder and depression 80\% \cite{judd1998comorbidity}, bipolar disorder and another psychological disorder 61\%\cite{raja2004clinical}. Comorbidity may reflect a common vulnerability underlying the several disorders that co-occurred. For instance, poor coping skills  \cite{andrews1996comorbidity} is one of the vulnerabilities that expose an individual to higher risks of anxiety and depressive disorders.   

In addition, DSM-5 stresses culture and gender related issues. In many cultures, somatic symptoms are predominant symptoms in depressive disorders. A majority of depressive disorder cases are unrecognized in primary care setting because sometimes cultural or gender related somatic symptoms are more predominant than affective symptoms. For example, a patient may complain about having headache, joint ache, catching a cold easily but the GP is not aware that these might be signs for affective disorders \cite{harvey2004cognitive}. 

\subsection{Symptoms}
Psychological disorders that have been studied in the social media context are mainly affective disorders because valence as an important symptoms in affective disorders can be identified in the social media text. In this section, we introduce some of the common symptoms in several types of affective disorders.
\textbf{Depressive disorders.} Symptoms occurring in most of the depressive disorders include: depressed mood, loss of interest or pleasure, insomnia or hypersonmnia, fatigue, anxiety, irritability and anger. How long does the depressive episode lasts distinguishes the type of depressive disorder \cite{american2013diagnostic}.

\textbf{Anxiety disorders.} The common symptoms in anxiety disorders include, excessive fear or anxiety concerning the theme, repeated nightmares involving the theme, repeated complaints of physical symptoms (e.g. headache, stomachache). For anxiety disorders, the fear, anxiety or avoidance are persistent \cite{american2013diagnostic}. The major distinction between different types of anxiety disorders lies in the themes or context that trigger the symptoms.

\textbf{Bipolar disorders.} Bipolar disorders include at least one manic episode and one major depressive episode.  During the manic episode, patient often has an inflated self-esteem, thought racing, mood swing, distractibility and decreased need for sleep. The major depressive episode is similar to the depressive episode in depressive disorders. The major distinction between different subcategories in bipolar disorder lies in the frequency of the manic-depressive cycles \cite{american2013diagnostic}.

\textbf{Stress and trauma related disorders.} Symptoms for trauma related disorders include, minimal social and emotional responsiveness to others, limited positive affect. Irritability, sadness or fearful of stimuli related to the stressful event. Intrusion symptoms such as recurrent memories or distressing dreams of the traumatic events. Symptoms that are specific to post-traumatic stress disorder (PTSD) include distorted cognition about the cause and consequences of the traumatic event and persistent negative emotional state. Loss of interests, irritability, hypervigilance, sleep disturbance, problem with concentration and so on \cite{american2013diagnostic}.

\subsection{Transdiagnostic framework and RDoC}
Transdiagnostic framework and RDoC are developed due to two reasons. On one hand, biological studies have shown that some disorders have similar brain structures and neurotransmitters. For example, amygdala is related to depression \cite{bowley2002low}, schizophrenia \cite{kosaka2002differential,kubicki2002voxel}, obsessive-compulsive disorder \cite{szeszko1999orbital}, and post-traumatic stress disorder \cite{rauch2000exaggerated}. On the other hand, the current diagnostic criteria is a categorical system that gives a clear cut-off line to disorders only based on the observation of symptoms. However, a lot of disorders have overlapped symptoms, which results in arbitrarily defined boundaries of the disorders. Hence, the current diagnostic criteria cannot explain the high ratio of comorbidity.

Recently, the emergence of transdiagnostic perspective understand psychological disorders outside of the traditional framework. It focuses on identifying the common factors that maintain the behaviors or symptoms of psychological disorders. The underlying constructs include attention, memory, reasoning, thought and behaviors \cite{harvey2004cognitive}. Transdiagnostic approaches to treatment refer to therapies that target a wide range of diagnosis without tailoring the protocol to specific diagnosis \cite{mansell2008cognitive}. A large amount of studies have shown the transdiagnostic approach is no less effective than traditional cognitive behavioral therapy (CBT) \cite{norton2008transdiagnostic,craske2012transdiagnostic,sauer2012role}.

Corresponding with the transdiagnostic treatment and research, in 2008, the US National Institute of Mental Health initiated the The Research Domain Criteria (RDoC) project in contrast to the Diagnostic and Statistical Manual of Mental Disorders in an attempt to link the current diagnostic criteria to the underlying neurobiological systems. The comorbidity of mental disorders might reflect that different patterns of symptoms have shared risk factors and underlying disease factors \cite{insel2010research}. The RDoC defines six constructs: negative valence systems (fear, anxiety, sustained threat, loss, frustrative nonreward), positive valence systems (approach motivation, initial responsiveness to reward, sustained responsiveness to reward, reward learning, habit), cognitive systems (attention, perception, declarative memory, working memory, cognitive control), systems for social processes (attachment formation, social communication, perception of self, perception of others), arousal/modulatory Systems (
arousal, circadian rhythm, sleep and wakefulness) \cite{insel2010research}. 

The most commonly studied constructs are negative valence system, positive valence system. Research on positive valence system focus on the reward-related constructs. Research on negative valence system examine at least one constructs in the five domains. Many of the studies employ a transdiagnostic approach examine a given constructs across multiple groups that are defined by clinical diagnosis. There are no social media studies that employ the transdiagnostic framework yet \cite{carcone2017six}. In chapter 4, we discuss identifying valence and thoughts on social media posts.

\subsection{symptoms and social media behavior}
There is a large amount of literature on using statistical prediction models to predict symptoms and diagnosis according to the DSM-5. For example, depression \cite{murrieta_depression:_2018,de2013social,reece2017forecasting} , PTSD \cite{reece2017forecasting}, Schizophrenia \cite{saha2017characterizing}, eating disorders \cite{chancellor2016quantifying} and so on. There are no studies that predict symptoms according to other diagnostic framework yet.

Social media platforms have various types of functionalities. Common functionalities include posts, comments, likes of the posts. We will discuss the differences between social media platforms in the next section. Some of the functionalities in social media platforms have been shown to be associated with symptomatic behaviours. Twitter users who have high score in self-reported depression measurement scale are found to have decreased social activity, more negative affect, highly clustered ego-networks and increased relational and medicinal concerns \cite{de2013predicting}. Mothers who developed postpartum disorders showed heightened social isolation and decreased social capital \cite{de2014characterizing}. User interests in advertisement predict Schizophrenia \cite{saha2017characterizing}. The shift of language pattern predicts suicidal risk \cite{de2016discovering}. Severity of eating disorder can be quantified by the posts' language \cite{chancellor2016quantifying}. 

Many studies also use the number of friends to infer the social support network. However, using social network size as a feature to infer social support could have a mixed effect because social ties can cause stress or provide social support. The number of comments in a post is often used as an indicator of social support in social media studies. In Chapter 3, we discuss number of comments might reflect some of the comments to the post are controversial or debatable. These comments do not reflect the amount of social support provided to the post author.


\section{Conclusion}
In this section, we introduce the 5 factors in case formulation model and we categories dozens of social media studies according to the 5 factors. Most predisposing factors cannot be inferred from social media data, for instance, the genetic component and family rearing style. Hence, a lot of studies focus on predicting personality or self-disclosure. Stress as a precipitating factor has been covered by numerous social media studies. Most of the social media studies on stress focus on analyzing people's reaction on acute stress and detecting life events. However, many social media studies on stress haven't connected stress with risk factors and protective factors. Although they are intertwined with stress, their interactions with stress has not yet been studied in the social media context.

Perpetuating factors is the least studied topic among social media literature. Chronic stress hasn't been widely explored yet. For example, stress from chronic illness and role stress as care taker of someone with a chronic condition. Another perpetuating factor that can be identified with social media data but hasn't been inspected yet is thoughts. Cognitive distortions are thoughts that indicative of a variety of mental health problems. Some of these thoughts can be identified in social media text. Chapter 4 discusses our approach to identify cognitive distortion on social media.

Protective factors may be positive or negative life experience, or may not be an experience at all. It could be the quality of a person, such as the predisposing factors (self-esteem, personality, self-disclosure) or it could be social support. All these factors interact with stress in predicting the risk of mental illness. Hence, the study of stress's effect on mental health can not be isolated from risk factors and protective factors. Among all the protective factors, social support is the most widely studied topic in social media context. There are a few gaps that have been identified: 1) most of these studies focused on enacted support, whereas, perceived support is overlooked. 2) Other components in protective factors, such as self-esteem, meaning making process and coping skills are not studied yet. 3) social media studies that mentioned social network often focus on the size of the network but overlooked the strength of ties. As a result, it is important to cover these gaps in the future studies.

For presenting problems, although numerous studies have shown that social media behaviours are related to mental health symptoms, yet the same pattern of change in social media behaviour is often found across various disorders. For example, users with depression, postpartum disorder, PTSD and other affective psychological disorders show increased usage of first person pronouns, swearing words and negation \cite{de_choudhury_major_2013,birnbaum_o9.2._2018}. However, the degree of quantified changes across different disorders hasn't been investigated yet. For example, inspecting whether the use of first person personal pronouns is more prominent among people with depression than among people with anxiety problem.  

The current symptom prediction models are all based on the DSM-5 framework, whereas, transdiagnostic framework is not explored yet in social media studies. We propose that instead of predicting specific symptoms of a certain disorder in the DSM-5 framework, future research can focus on predicting symptoms that occur across various types of disorders. In chapter 4, we discuss predicting the cognition symptoms in social media context.


-------
%Conclusion of literature review 
%A New Dimension of Health Care: Systematic Review of the Uses, Benefits, and Limitations of Social Media for Health Communication

%the key recommendations for future health communication research focus on robust and comprehensive evaluation and review, using a range of methodologies. The research priorities are highlighted below:

%To determine the impact of social media for health communication in specific population groups with large sample sizes (representation of population groups).
%To determine the relative effectiveness of different social media applications for health communication using RCTs.
%To determine the longer-term impact on the effectiveness of social media for health communication using longitudinal studies.
%To explore potential mechanisms for monitoring and enhancing the quality and reliability of health communication using social media.
%To investigate the risks arising from sharing information online and the consequences for confidentiality and privacy, coupled with developing the most suitable mechanisms to effectively educate users in the maintenance of their confidentiality and privacy.
%To determine how social media can be effectively used to support the patient-health professional relationship.
%To determine the impact of peer-to-peer support for the general public, patients, and health professionals to enhance their interpersonal communication.
%To explore the potential for social media to lead to behavior change for healthy lifestyles to inform health communication practice.
%(A New Dimension of Health Care: Systematic Review of the Uses, Benefits, and Limitations of Social Media for Health Communication)


